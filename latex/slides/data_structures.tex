\def\blocksize{12mm}

\section{Arrays vs Linked Lists}
\begin{frame}
    \frametitle{Arrays vs Linked Lists}

    \begin{figure}[h]
    \centering
    \begin{tikzpicture}[%
        font=\small\ttfamily,
        x=\blocksize, y=\blocksize,
        block/.style={draw,minimum size=\blocksize,inner sep=0pt}
        ]
        \tikzset{used/.style={block,fill=red!30,draw=black!50}}
        \tikzset{free/.style={block,fill=black!20,draw=black!50}}

        \node[free,label=below:\scriptsize 0x09F] (c-1) at (0,0) {};
        \foreach \l [count=\i from 0] in {100,...,105}
            \node[used,label=below:\scriptsize 0x\l] (c\i) at (\i+1,0)
                { arr[\i] };
        \foreach \l [count=\i from 7] in {106,...,107}
            \node[free,label=below:\scriptsize 0x\l] (c\i) at (\i,0) {};
        \node[block,thick,fit=(c-1)(c8)] {};

    \end{tikzpicture}
    \caption{Array}
    \end{figure}

    \textbf{Arrays}
    \begin{itemize}
        \item represents the computers memory
        \item random access (i.e. indexing)
        \item fixed in size and order
    \end{itemize}

\end{frame}


\begin{frame}[fragile]

    \begin{figure}[h]
    \centering
    \begin{tikzpicture}[%
        font=\small\ttfamily,
        list/.style={%
            draw,minimum size=\blocksize,
            rectangle split,
            rectangle split parts=3,
            rectangle split horizontal,
            fill=red!30
        },
        ->, start chain, thick
        ]

        \node[list,on chain] (A) {\nodepart{second} 1st node};
        \node[list,on chain] (B) {\nodepart{second} 2nd node};
        \node[list,on chain] (C) {\nodepart{second} 3rd node};

        \path[*->] let \p1 = (A.three), \p2 = (A.center) in (\x1,\y2) edge [bend left] ($(B.one)+(0,0.2)$);
        \path[*->] let \p1 = (B.three), \p2 = (B.center) in (\x1,\y2) edge [bend left] ($(C.one)+(0,0.2)$);

        \path[*->] ($(B.one)+(0.2,0.1)$) edge [bend left] ($(A.three)+(0,-0.05)$);
        \path[*->] ($(C.one)+(0.1,0.1)$) edge [bend left] ($(B.three)+(0,-0.05)$);

    \end{tikzpicture}
    \caption{Doubly Linked List}
    \end{figure}

    \begin{columns}[c]
        \column{.5\textwidth}
        \textbf{Linked Lists}
        \begin{itemize}
            \item independent of memory layout
            \item very flexible
            \item extra space needed to store adresses of predecessor and
                successor
        \end{itemize}

        \column{.5\textwidth}
        \textit{Implementation Example}
        \begin{minted}[linenos,tabsize=4]{c}
struct node_t {
    node_t *prev, *next;
    void *data;
};
        \end{minted}
    \end{columns}

\end{frame}

% vim: set ff=unix tw=79 sw=4 ts=4 et ic ai :
